\documentclass{article}
\usepackage{amsmath,amsthm,amssymb}
\usepackage{verified-badges}

\newtheorem{theorem}{Theorem}[section]
\newtheorem{lemma}[theorem]{Lemma}
\newtheorem{proposition}[theorem]{Proposition}
\newtheorem{definition}[theorem]{Definition}

\title{Example: Verified Badges}
\author{}
\date{}

\begin{document}
\maketitle

\section{Results}

\begin{theorem}[Confluence \leanproof{https://github.com/example/proofs/blob/main/Confluence.lean}]
  If $R$ is a terminating abstract reduction system satisfying the diamond
  property, then $R$ is confluent.
\end{theorem}

\begin{lemma}[\coqproof{https://github.com/example/proofs/blob/main/Diamond.v}]
  If $\to$ satisfies the diamond property, then $\twoheadrightarrow$
  satisfies the Church-Rosser property.
\end{lemma}

\begin{proposition}[Normalization \isabelleproof{https://github.com/example/proofs/blob/main/Normalization.thy}]
  Every well-typed term in System~F has a normal form.
\end{proposition}

\begin{definition}[Abstract reduction system \leanformalized{https://github.com/example/proofs/blob/main/ARS.lean}]
  An \emph{abstract reduction system} is a pair $(A, \to)$ where $A$ is a set
  and ${\to} \subseteq A \times A$.
\end{definition}

\begin{lemma}[Diamond property \leanformalized{https://github.com/example/proofs/blob/main/Diamond.lean}]
  The diamond property implies the Church-Rosser property.
  (Statement formalized; proof in progress.)
\end{lemma}

\begin{theorem}
  Not every result has been formally verified yet.
\end{theorem}

\end{document}
